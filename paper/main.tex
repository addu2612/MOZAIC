\documentclass[conference]{IEEEtran}

% --- Packages ---
\usepackage{graphicx}
\usepackage{booktabs}
\usepackage{amsmath}
\usepackage{hyperref}
\usepackage{caption}

\title{MOZAIC: Multi-Source Orchestrated Zephyr Anomaly Intelligent Coordinator}

\author{\IEEEauthorblockN{Author Name(s)}
\IEEEauthorblockA{Affiliation \\ Email}}

\begin{document}
\maketitle

\begin{abstract}
% TODO: Write abstract.
This paper presents MOZAIC, a full-stack system for ingesting monitoring signals from multiple sources, clustering and correlating them into incidents, and presenting them via a web interface. We describe the architecture, orchestration pipeline, and evaluation plan.
\end{abstract}

\begin{IEEEkeywords}
observability, incident management, log clustering, correlation, orchestration
\end{IEEEkeywords}

\section{Introduction}
% TODO: motivation + contributions.

\section{System Overview}
% TODO: describe frontend/backend, ingestion sources, data model.

\section{Orchestration Pipeline}
% TODO: how data is fetched, normalized, clustered, correlated.

\section{Clustering and Correlation Methods}
% TODO: baseline vs proposed embedding-based clustering.

\section{Experimental Setup}
% TODO: datasets, evaluation metrics, baselines.

\section{Results}
% TODO: add quantitative tables/plots once available.

\begin{table}[t]
  \caption{Placeholder Results Table}
  \label{tab:results}
  \centering
  \begin{tabular}{lccc}
    \toprule
    Method & Precision & Recall & F1 \\
    \midrule
    Baseline & -- & -- & -- \\
    MOZAIC (proposed) & -- & -- & -- \\
    \bottomrule
  \end{tabular}
\end{table}

\begin{figure}[t]
  \centering
  % TODO: replace with an actual PDF/PNG in paper/figures/
  \fbox{\parbox[c][3cm][c]{0.9\linewidth}{\centering Placeholder Figure}}
  \caption{Placeholder for architecture or results figure.}
  \label{fig:placeholder}
\end{figure}

\section{Discussion}
% TODO: limitations, failure modes, operational constraints.

\section{Conclusion and Future Work}
% TODO: summarize and outline next steps.

\section*{Acknowledgments}
% TODO: optional.

\bibliographystyle{IEEEtran}
\bibliography{refs}

\end{document}
